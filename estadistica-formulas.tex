% !TEX root = formulas-estadisticas.tex 
% Author: Alfredo Sánchez Alberca (asalber@ceu.es)

\sloppy

\section*{Formulas de Estadística}

\footnotesize
\tcbset{enhanced, colback=color1!10, colframe=color1, fonttitle=\bfseries\large\sffamily}

\begin{multicols*}{2}

\subsection*{Estadística Descriptiva}

\begin{tcolorbox}[hbox, title=Frecuencias]
\begin{minipage}{0.4\textwidth}
\begin{description}
\item [Tamaño muestral] $n$ número de individuos en la muestra.
\end{description}
\begin{description}
\item [Frecuencia Absolulta] $n_i$ (nº de $x_i$ en la muestra)
\item [Frecuencia Relativa] $f_i=n_i/n$
\item [Frec. Absoluta Acumulada] $N_i=\sum_{k=0}^in_i$
\item [Frec. Relativa Acumulada] $F_i=N_i/n$
\end{description}
\end{minipage}
\end{tcolorbox}

\begin{tcolorbox}[hbox, title=Estadísticos de tendencia central]
\begin{minipage}{0.4\textwidth}
\begin{description}
\item [Media] $\bar{x}=\dfrac{\sum x_i}{n}$
\item [Mediana] $me$ El valor con frec. rel. acumulada $F_{me}=0.5$.
\item [Moda] $mo$ El valor más frecuente.
\end{description}
\end{minipage}
\end{tcolorbox}

\begin{tcolorbox}[hbox, title=Estadísticos de posición]
\begin{minipage}{0.4\textwidth}
\begin{description}
\item [Cuartiles] $Q_1,Q_2,Q_3$ dividen la distribución en 4 partes iguales.
      Sus frec. rel. acumuladas son
      $F_{Q_1}=0.25$, $F_{Q_2}=0.5$ and $F_{Q_3}=0.75$.
\item [Percentiles] $P_1,P_2,\cdots,P_{99}$ dividen la distribución en 100 partes iguales.\\
      Su frec. rel. acumulada es $F_{P_i}=i/100$.
\end{description}
\textbf{Interpolación}

\resizebox{\textwidth}{!}{% Author: Alfredo Sánchez Alberca (asalber@ceu.es)

\pgfplotsset{
    standard/.style={
        axis x line=middle,
        axis y line=middle,
        % enlarge x limits=0.15,
        % enlarge y limits=0.15,
        every axis x label/.style={at={(current axis.right of origin)},anchor=north west},
        every axis y label/.style={at={(current axis.above origin)},anchor=north east}
    }
}

\begin{tikzpicture}
\begin{axis}[standard, xlabel={$X$}, ylabel={$F$}, axis equal, xmin=-0.1, xmax=7, ymin=0,
ymax=5, xtick={1,6}, xticklabels={$l_{i-1}$,$l_i$}, ytick={1,4}, yticklabels={$F_{i-1}$,$F_i$}]

\coordinate (A) at (axis cs:6,1);
\coordinate (B) at (axis cs:1,1);
\coordinate (C) at (axis cs:6,4);
\coordinate (D) at (axis cs:4,1);
\coordinate (E) at (axis cs:4,2.8);
\coordinate (F) at (axis cs:-0.1,2.8);
\coordinate (G) at (axis cs:4,-0.1);
\end{axis}

\draw (B) -- (C);

\draw[fill=color1!20] (A) -- (B) -- (C) -- cycle;
\draw[fill=color2!20] (D) -- (B) -- (E) -- cycle;

\tkzMarkAngle[fill= green!50,size=1cm](A,B,C)
\tkzLabelAngle[pos = 0.7](A,B,C){$\alpha$}
%\node[anchor=west] at (7,3) {$\color{color1} \displaystyle \tan(\alpha) = \frac{F_i-F_{i-1}}{l_i-l_{i-1}}$};

\node[anchor=east] at (F) {$\frac{i}{100}$};
\draw[dashed] (F) -- (E) -- (G);
\node[anchor=north] at (G) {\color{color2}$P_i$};

%\node[anchor=west] at (7,2) {$\color{color2} \displaystyle \tan(\alpha) = \frac{0.5-F_{i-1}}{Me-l_{i-1}}$};


\end{tikzpicture}}

\[P_i=l_i+\frac{\frac{i}{100}-F_{i-1}}{F_i-F_{i-1}}(l_i-l_{i-1})\]
\end{minipage}
\end{tcolorbox}

\begin{tcolorbox}[hbox, title=Estadísticos de dispersión]
\begin{minipage}{0.4\textwidth}
\begin{description}
\item [Rango intercuartílico] $IQR=Q_3-Q_1$
\item [Varianza] $s^2=\dfrac{\sum (x_i-\bar x)^2}{n}=\dfrac{\sum x_i^2}{n}-\bar x^2$
\item [Desviación típica] $s=+\sqrt{s^2}$
\item [Coeficiente de variación] $cv=\dfrac{s}{|\bar{x}|}$
\end{description}
\end{minipage}
\end{tcolorbox}

\begin{tcolorbox}[hbox, title=Estadísticos de forma]
\begin{minipage}{0.4\textwidth}
\begin{description}
\item [Coeficiente de asimetría] $g_1=\dfrac{\sum(x_i-\bar{x})^3}{ns^3}$
\item [Coeficiente de apuntamiento] $g_2=\dfrac{\sum(x_i-\bar{x})^4}{ns^4}-3$
\end{description}
\end{minipage}
\end{tcolorbox}

\begin{tcolorbox}[hbox, title=Transformaciones lineales]
\begin{minipage}{0.4\textwidth}
\begin{description}
\item[Transformación lineal] $y=a+bx$
      \begin{align*}
      \bar y & = a+b\bar x \\
      s_y    & = bs_x
      \end{align*}
\item[Tipificación] $z=\dfrac{x-\bar x}{s_x}$
\end{description}
\end{minipage}
\end{tcolorbox}


\subsection*{Regresión y correlación}

\begin{tcolorbox}[hbox, title=Regresión lineal]
\begin{minipage}{0.4\textwidth}
\begin{description}
\item [Covarianza] $s_{xy}=\dfrac{\sum x_iy_j}{n}-\bar{x}\bar{y}$
\item [Rectas de regresión]:
      \begin{align*}
      \mbox{$y$ on $x$} & : y=\bar{y}+\dfrac{s_{xy}}{s_x^2}(x-\bar{x}) \\
      \mbox{$x$ on $y$} & : x=\bar{x}+\dfrac{s_{xy}}{s_y^2}(y-\bar{y})
      \end{align*}
\item [Coeficientes de regresión]
      \[
      \mbox{($y$ on $x$) } b_{yx}=\dfrac{s_{xy}}{s_x^2}\quad \mbox{($x$ on
      $y$) } b_{xy}=\dfrac{s_{xy}}{s_y^2}
      \]
\item[Coeficiente de determinación]
      \[r^2=\dfrac{s_{xy}^2}{s_x^2s_y^2} \qquad 0\leq r^2\leq 1\]
\item[Coeficiente de correlación]
      \[r=\dfrac{s_{xy}}{s_xs_y}.\qquad -1\leq r\leq 1\]
\end{description}
\end{minipage}
\end{tcolorbox}

\medskip

\begin{tcolorbox}[hbox, title=Regresión no lineal]
\begin{minipage}{0.4\textwidth}
\begin{description}
\item[Modelo exponencial] $y=e^{a+bx}$\\
      Aplicar el logaritmo a la variable dependiente y calcular la recta de regresión $\log y = a+bx$.
\item[Modelo logarítmico] $y=a+b\log x$\\
      Aplicar el logaritmo a la variable independiente y calcular la recta de regresión $y=a+b\log x$.
\item[Modelo potencial] $y=ax^b$\\
      Aplicar el logaritmo a ambas variables y calcular la recta de regresión $\log y = a+b\log x$.
\end{description}
\end{minipage}
\end{tcolorbox}

\newpage

\subsection*{Probabilidad}

\begin{tcolorbox}[hbox, title=Operaciones de sucesos]
\begin{minipage}{0.4\textwidth}
\textbf{Unión}
\begin{center}
% Author: Alfredo Sánchez Alberca (asalber@ceu.es)
\begin{tikzpicture}
\def\firstcircle{(1.5,1.5) circle (1cm)}
\def\secondcircle{(2.5,1.5) circle (1cm)}

\fill[color1!30] \firstcircle;
\fill[color1!30] \secondcircle;
\draw (0,3) node[anchor=north east] {$\Omega$} rectangle (4,0);
\draw \firstcircle node[xshift=-0.9cm, yshift=0.9cm] {$A$};
\draw \secondcircle node[xshift=0.9cm, yshift=0.9cm] {$B$};

\node at (2,0.3) {$A\cup B$};
\end{tikzpicture}
\end{center}
\textbf{Intersección}
\begin{center}
% Author: Alfredo Sánchez Alberca (asalber@ceu.es)

\begin{tikzpicture}
\def\firstcircle{(1.5,1.5) circle (1cm)}
\def\secondcircle{(2.5,1.5) circle (1cm)}

\begin{scope}
\clip \firstcircle;
\fill[color1!30] \secondcircle;
\end{scope}

\draw (0,3) node[anchor=north east] {$\Omega$} rectangle (4,0);
\draw \firstcircle node[xshift=-0.9cm, yshift=0.9cm] {$A$};
\draw \secondcircle node[xshift=0.9cm, yshift=0.9cm] {$B$};

\node at (2,1.5) {$A\cap B$};
\end{tikzpicture}
\end{center}
\textbf{Contrario}
\begin{center}
% Author: Alfredo Sánchez Alberca (asalber@ceu.es)

\begin{tikzpicture}
\def\circle{(1.5,1.5) circle (1cm)}
\def\rectangle{(4,0) rectangle (0,3)}

\begin{scope}[even odd rule]
\clip \circle (0,0) rectangle (4,3);
\fill[color1!30] \rectangle;
\end{scope}

\draw \rectangle node[anchor=north east] {$\Omega$};
\draw \circle node {$A$};
\node at (3,1.5) {$\overline A$};
\end{tikzpicture}
\end{center}
\textbf{Diferencia}
\begin{center}
% Author: Alfredo Sánchez Alberca (asalber@ceu.es)

\begin{tikzpicture}
\def\firstcircle{(1.5,1.5) circle (1cm)}
\def\secondcircle{(2.5,1.5) circle (1cm)}

\begin{scope}[even odd rule]
\clip \secondcircle (0,0) rectangle (4,3);
\fill[color1!30] \firstcircle;
\end{scope}

\draw (0,3) node[anchor=north east] {$\Omega$} rectangle (4,0);
\draw \firstcircle node[xshift=-0.9cm, yshift=0.9cm] {$A$};
\draw \secondcircle node[xshift=0.9cm, yshift=0.9cm] {$B$};

\node[anchor=east] at (1.5,1.5) {$A-B$};
\end{tikzpicture}
\end{center}
\end{minipage}
\end{tcolorbox}

\begin{tcolorbox}[hbox, title=Álgebra de sucesos]
\begin{minipage}{0.4\textwidth}
\begin{description}
\item[Idempotencia] $A\cup A=A$,\quad $A\cap A=A$
\item[Conmutativa] $A\cup B=B\cup A$,\quad $A\cap B = B\cap A$
\item[Asociativa] $(A\cup B)\cup C = A\cup (B\cup C)$,\quad $(A\cap B)\cap C = A\cap (B\cap C)$
\item[Distributiva] $(A\cup B)\cap C = (A\cap C)\cup (B\cap C)$,\quad $(A\cap B)\cup C = (A\cup C)\cap (B\cup C)$
\item[Elemento neutro] $A\cup \emptyset=A$,\quad $A\cap \Omega=A$
\item[Elemento absorvente] $A\cup \Omega=\Omega$,\quad $A\cap \emptyset=\emptyset$.
\item[Elemento simétrico complementario] $A\cup \overline A = \Omega$,\quad $A\cap \overline A= \emptyset$
\item[Doble contrario] $\overline{\overline A} = A$
\item[Leyes de Morgan] $\overline{A\cup B} = \overline A\cap \overline B$,\quad $\overline{A\cap B} = \overline A\cup \overline B$
\end{description}
\end{minipage}
\end{tcolorbox}

\begin{tcolorbox}[hbox, title=Probabilidad básica]
\begin{minipage}{0.4\textwidth}
\begin{description}
\item [Unión] $P(A\cup B)=P(A)+P(B)-P(A\cap B)$
\item [Intersección] $P(A\cap B)=P(A)P(B|A)$
\item [Diferencia] $P(A-B)=P(A)-P(A\cap B)$
\item [Contrario] $P(\overline{A})=1-P(A)$
\end{description}
\end{minipage}
\end{tcolorbox}

\begin{tcolorbox}[hbox, title=Probabilidad condicionada]
\begin{minipage}{0.4\textwidth}
\begin{description}
\item [Probabilidad condicionada] $P(A|B)=\dfrac{P(A\cap B)}{P(B)}$
\item [Sucesos independientes] $P(A|B)=P(A)$.
\item [Teorema de la probabilidad total] \[P(B)=\sum_{i=1}^n P(A_i)P(B|A_i)\]
\item [Teorema de Bayes] \[P(A_i|B)=\dfrac{P(A_i)P(B|A_i)}{\sum_{i=1}^n P(A_i)P(B|A_i)}\]
\end{description}
\end{minipage}
\end{tcolorbox}

\begin{tcolorbox}[hbox, title=Riesgos]
\begin{minipage}{0.4\textwidth}
\begin{center}
\begin{tabular}{|l|c|c|}
\cline{2-3}
\multicolumn{1}{c|}{} & $E$ & $\overline E$ \\
\hline
Tratamiento           & $a$ & $b$           \\
\hline
Control               & $c$ & $d$           \\
\hline
\end{tabular}
\end{center}
\begin{description}
\item[Prevalencia] Proporción de individuos con el suceso $E$: $P(E)$
\item[Tasa de incidencia o riesgo absoluto] $R(E)=\dfrac{a}{a+b}$
\item[Odds] $O(E)=\dfrac{a}{b}$
\item[Riesgo relativo] $RR(E)=\dfrac{a/(a+b)}{c/(c+d)}$
\item[Odds ratio] $OR(E)=\dfrac{a/b}{c/d}=\dfrac{a\cdot d}{b\cdot c}$
\end{description}
\end{minipage}
\end{tcolorbox}


\begin{tcolorbox}[hbox, title=Test diagnósticos]
\begin{minipage}{0.4\textwidth}
\begin{center}
\begin{tabular}{|l|c|c|}
\cline{2-3}
\multicolumn{1}{c|}{} & Enfermo $E$ & Sano $\overline E$ \\
\hline
Test $+$              & $VP$    & $FP$ \\
\hline
Test $-$              & $FN$    & $VN$ \\
\hline
\end{tabular}
\end{center}
\begin{description}
\item[Sensibilidad] $P(+|E)=\dfrac{VP}{VP+FN}$
\item[Especificidad] $P(-|\overline{E})=\dfrac{VN}{FP+VN}$
\item[Valor predictivo positivo (VPP)] $P(E|+)=\dfrac{VP}{VP+FP}$
\item[Valor predictivo negativo (VPN)] $P(\overline{E}|-)=\dfrac{VN}{FN+VN}$
\item[Razón de verosimilitud positiva (RV+)] $\dfrac{P(+|E)}{P(+|\overline{E})}$
\item[Razón de verosimilitud negativa (RV-)] $\dfrac{P(-|E)}{P(-|\overline{E})}$
\end{description}
\end{minipage}
\end{tcolorbox}


\subsection*{Variables Aleatorias}

\begin{tcolorbox}[hbox, title=Discretas]
\begin{minipage}{0.4\textwidth}
\begin{description}
\item [Función de probabilidad Binomial $B(n,p)$]
      \[f(x)=\binom{n}{x}p^x (1-p)^{n-x}=\dfrac{n!}{x!(n-x)!}p^x (1-p)^{n-x}\]
\item [Función de probabilidad Poisson $P(\lambda)$]
      \[f(x)=e^{-\lambda}\frac{\lambda^x}{x!}\]
\item [Ley de los casos raros] $B(n,p)\approx P(np)$ para $n\geq 30$ y $p\leq 0.1$.
\end{description}
\end{minipage}
\end{tcolorbox}

\begin{tcolorbox}[hbox, title=Continuas]
\begin{minipage}{0.4\textwidth}
\begin{description}
\item[Normal $N(\mu,\sigma)$]
      \[f(x)= \frac{1}{\sigma\sqrt{2\pi}}e^{-\frac{(x-\mu)^2}{2\sigma^2}}\]
      \textbf{Normal Estándar $N(0,1)$}
\item[Chi-cuadrado $\chi^2(n)$]
      \[X = Z_1^2+\cdots +Z_n^2,\]
      donde $Z_i\sim N(0,1)$.
\item[T de Student $T(n)$]
      \[T = \frac{Z}{\sqrt{X/n}},\]
      donde $Z\sim N(0,1)$ y $X\sim \chi^2(n)$.
\item[F de Fisher $F(n,m)$]
      \[F = \frac{X/m}{Y/n},\]
      donde $X\sim \chi^2(m)$ y $Y\sim \chi^2(n)$.
\end{description}
\end{minipage}
\end{tcolorbox}

\end{multicols*}