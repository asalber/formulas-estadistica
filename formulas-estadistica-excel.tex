% !TEX program = xelatex
% Author: Alfredo Sánchez Alberca (asalber@ceu.es)
\documentclass[a4paper, 10pt, landscape]{article}
\usepackage[no-math]{fontspec}
\setmainfont[BoldFont={Fira Sans}]{Fira Sans Light}
\setmonofont{Fira Mono}
\usepackage[left=2cm,right=2cm,top=2cm,bottom=2cm]{geometry}
\usepackage{xcolor}
\definecolor{blueceu}{RGB}{5,161,230}
\usepackage{array}
\usepackage{graphicx}
\usepackage{ragged2e}
\usepackage{fancyhdr}
\usepackage{hyperref}
\hypersetup{colorlinks=true}
\pagestyle{fancy}
\rhead{\url{http://aprendeconalf.es}}
\renewcommand{\headrulewidth}{0pt}


\begin{document}
\section*{\color{blueceu}Statistical formulas}
\scalebox{0.92}{
\begin{tabular}{| >{\RaggedRight}l | >{\RaggedRight}l |
>{\centering\arraybackslash}c | >{\centering\arraybackslash}c | >{\RaggedRight}m{8cm}|}
\multicolumn{2}{c}{} & \multicolumn{2}{c}{\bf Formulas} &
\multicolumn{1}{c}{}\\
\hline
\bf Statistics & \bf Excel function & \bf Sample & \bf Freq. table & \bf Interpretation \\
\hline\hline
Mean $\bar x$ & \tt AVERAGE(sample) & $\displaystyle \frac{\sum x_i}{n}$ & $\sum{x_if_i}$ & The value that best
represent the values of the sample (except when there are outliers).\\
\hline
Median $Me$ & \tt MEDIAN(sample) & & & The value in the middle of the ordered sample. 50\% of values of the sample are
above and 50\% below it.\\
\hline
Mode $Mo$ & \tt MODE(sample) & & & The most common value in the sample.\\
\hline
Minimum $Min$ & \tt MIN(sample) & $\min\{x_i\}$ & & The minimum value of the sample.\\
\hline
Maximum $Max$ & \tt MAX(sample) & $\max\{x_i\}$ & & The maximum value of the sample.\\
\hline
First quartile $Q_1$ & \tt QUARTILE(sample,1) & & & 25\% of the values of the sample are lower or equal to it.\\
\hline
Second quartile $Q_2$ & \tt QUARTILE(sample,2) & & & 50\% of the values of the sample are lower or equal to it.\\
\hline
Third quartile $Q_3$ & \tt QUARTILE(sample,3) & & & 75\% of the values of the sample are lower or equal to it.\\
\hline
Decile $i$ $D_i$ & \tt PERCENTILE(sample, i/10) & & & $i*10$\% of the values of the sample are lower or equal to it.\\
\hline
Percentile $i$ $P_i$ & \tt PERCENTILE(sample, i/100) & & & $i$\% of the values of the sample are lower or equal to it.\\
\hline
Range & \tt MAX(sample)-MIN(sample) & $Max-Min$ & $Max-Min$ & Measures the overall spread of the sample.\\
\hline
Interquartile Range $IQR$ &\tt  QUARTILE(sample,3)-QUARTILE(sample,1) & $Q_3-Q_1$ & $Q_3-Q_1$ & Measures the spread of
the 50\% central values of the sample.\\
\hline
Variance $s^2$ & \tt VAR.P(sample) & $\displaystyle \frac{\sum (x_i-\bar x)^2}{n}$ & $\sum (x_i-\bar x)^2f_i$ & Measures
the average spread with respect to the mean in square units.\\
\hline
Standard deviation $s$ & \tt STDEV.P(sample) & $\sqrt{s^2}$ & $\sqrt{s^2}$ & Measures the average spread with respect to
the mean in the units of the variable. \\
\hline
Coef. variation $cv$ & \tt STDEV.P(sample)/ABS(AVERAGE(SAMPLE)) & $\displaystyle \frac{s}{|\bar x|}$ & $\displaystyle
\frac{s}{|\bar x|}$ & Measures the relative spread with respect to the mean. It has no units. The lower the dispersion,
the more representative is the mean.\\
\hline
Coef. skewness $g_1$ & \tt SKEW(sample) & $\displaystyle \sum \left(\frac{x_i-\bar x}{s}\right)^3$ & $\displaystyle
\sum \left(\frac{x_i-\bar x}{s}\right)^3 f_i$ & Measures the asymmetry of the sample distribution ($g_1=0$ symmetry,
$g_1>0$ right-skewed, $g_1<0$ left-skewed).\\
\hline
Coef. kurtosis $g_2$ & \tt KURT(sample) & $\displaystyle \sum \left(\frac{x_i-\bar x}{s}\right)^4 -3$ & $\displaystyle
\sum \left(\frac{x_i-\bar x}{s}\right)^4 f_i -3$ & Measures the peakness or flatness of the sample distribution
compared to a normal distribution ($g_2=0$ normal kurtosis, $g_1>0$ leptokurtic or peaked distribution, $g_1<0$
platykurtic or flat distribution).\\
\hline
\end{tabular}
}
\end{document}
